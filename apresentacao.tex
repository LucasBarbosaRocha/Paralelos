%!TEX program = xelatex
\documentclass[10pt, compress]{beamer}
\usetheme[titleprogressbar]{m}

\usepackage[portuguese]{babel}  
\usepackage{booktabs}
\usepackage[scale=2]{ccicons}
\usepackage{minted}

\newcommand{\aaa}{A}
\newcommand{\gga}{G}
\newcommand{\ccc}{C}
\newcommand{\ttt}{T}
%\newcommand{\ttt}{\mathtt{T}}

\usepgfplotslibrary{dateplot}
\usemintedstyle{trac}

\title{O Problema da Seleção de Primers Específicos usando Distância de Hamming}
\subtitle{}
\author{Aluno: Lucas B. Rocha \newline Orientadores: Francisco Elói e Said Sadique}

\institute{Universidade Federal de Mato Grosso do Sul}

\begin{document}

\maketitle

\begin{frame}[fragile]
	\frametitle{Introdução}
    
    \alert{Motivações}
    
    \begin{itemize}
    	\item Problemas com sequências; \pause
        \item Detecção de doenças; \pause
        \item Identificar regiões específicas de DNA para amplificar; \pause
        \item Técnica PCR.
    \end{itemize}
    
%\begin{figure}
%  \centering
% \includegraphics[width=6cm,height=6cm]{firstex.png}
%\end{figure}
\end{frame}

\begin{frame}[fragile]
	\frametitle{Problema}
  	\begin{itemize}
  		\item \textbf{Biólogo}: Detectar doenças ou agentes causadores de doenças;
        \item \textbf{Como}: Identificando regiões específicas de DNA;
        \item \textbf{Solução}: Dadas duas sequências, analisar todos os seguimentos possíveis na busca de um que não aparece na outra.
        \item \textbf{Dificuldade}: As sequências são grandes, e a identificação dessas regiões consomem muito tempo;
	\end{itemize}
\end{frame}

\begin{frame}[fragile]
	\frametitle{Conceitos}
    \begin{itemize}
        \item \textbf{Primer}:  Oligonucleotídeos sintéticos curtos, ou seja, é um fragmento curto de uma cadeia simples de ácido nucleico (DNA ou RNA), tipicamente com 20 ou menos bases.
        
        \vspace{0.5cm} \pause
        
        \item Exemplo de característica do primer ideal:
        	\begin{itemize}
        		\item Pode ser selecionado de qualquer ponto da sequência molde;
				\item Ser o menor segmento possível, dado que o custo da sintetização do primer é
diretamente proporcional ao tamanho do segmento; assim, quanto maior, mais
caro.
			\end{itemize}
    \end{itemize}
\end{frame}

\begin{frame}[fragile]
	\frametitle{Conceitos}
    \begin{itemize}
    	\item \textbf{Sequência}: $\sum = \{A, C, G, T\}$, $s = TTAGCGAC$,  $|s| = 8$, e $s_2 = T$
		\item $\sum = $ \textbf{DNA}: (\textbf{A}) adenina, (\textbf{C}) citosina, (\textbf{G}) guanina e (\textbf{T}) timina.
        
        \vspace{0.5cm} \pause
        
        \item $s_h$ := $h$-ésimo elemento de $s$, 
        
        \item \textbf{Subcadeia}: $s_{i \ldots j} = s_i \ s_{i+1} \ \ldots \ s_j$.
	    % $j$ é p $j$-ésimo último elemento. Para $i < j$ a cadeia é vazia $(\epsilon)$.
		\item \textbf{Exemplo}: $s_{2 \ldots 5} = TAGC$.
    \end{itemize}
\end{frame}

\begin{frame}[fragile]
	\frametitle{Conceitos}
	Distância de Edição:
    \begin{itemize}
    	\item \textbf{Entrada}: Duas sequências $\alpha$ e $\beta$;
        \item \textbf{Objetivo}: O número mínimo de operações para transformar $\alpha$ em $\beta$ ou vice-versa.
        \item \textbf{Operações}: \textbf{I}nserção, \textbf{R}emoção e \textbf{S}ubstituição.
    \end{itemize}
\end{frame}

\begin{frame}[fragile]
	\frametitle{Conceitos}
	Distância de Edição - Exemplo:
    \begin{itemize}
    	\item $\alpha = TACATG$ e $\beta = ACAACT$
        \begin{itemize}
        	\item \textbf{R.} $\alpha_1 = \alert{T}ACATG = ACATG$
            \item \textbf{S.} $\alpha_5 \ por \ A = ACA\alert{T}G = ACAAG$
            \item \textbf{S.} $\alpha_6 \ por \ C = ACAA\alert{G} = ACAAC$
            \item \textbf{I.} $T \ no \ final = ACAACT$
        \end{itemize}
        \item $D(\alpha,\beta) = 4$
    \end{itemize}
\end{frame}

\begin{frame}[fragile]
	\frametitle{Conceitos}
	Distância de Hamming:
    \begin{itemize}
    	\item \textbf{Entrada}: Duas sequências $\alpha$ e $\beta$;
        \item \textbf{Objetivo}: $dist(\alpha,\beta) = \sum_{1 \le i \le n} dif(i)$,
        					$$
							dif(i,j) = \begin{cases}
											0 & \mbox{se $\alpha_i = \beta_j$,} \\
                    						1 & \mbox{caso contrário}.
										\end{cases}
							$$
    \end{itemize}
    Exemplo:
    \begin{itemize}
    	\item $\alpha = TACATG$ e $\beta = ACAACT$
    \end{itemize}
    \begin{tikzpicture}
    	\path 
    	node at (0,0)    {$\ttt$}
    	node at (0.3,0)  {$\aaa$}
    	node at (0.6,0)  {$\ccc$}
    	node at (0.9,0)  {$\aaa$}
    	node at (1.2,0)  {$\ttt$}
    	node at (1.5,0)  {$\gga$};
        \path        
    	node at (0,-0.5)   {$\aaa$}
    	node at (0.3,-0.5) {$\ccc$}
    	node at (0.6,-0.5) {$\aaa$}
    	node at (0.9,-0.5)  {$\aaa$}
    	node at (1.2,-0.5) {$\ccc$}
    	node at (1.5,-0.5) {$\ttt$};
        \pause
    	\path 
    	node at (0,0)    {$\alert{\ttt}$};
        \path        
    	node at (0,-0.5)   {$\alert{\aaa}$}; 
        \pause
    	\path 
    	node at (0.3,0)  {$\alert{\aaa}$};
        \path        
    	node at (0.3,-0.5) {$\alert{\ccc}$};
        \pause
    	\path 
    	node at (0.6,0)  {$\alert{\ccc}$};
        \path        
    	node at (0.6,-0.5) {$\alert{\aaa}$};
        \pause
    	\path 
    	node at (0.9,0)  {$\textcolor{blue}{\aaa}$};
        \path        
    	node at (0.9,-0.5) {$\textcolor{blue}{\aaa}$};
        \pause
    	\path 
    	node at (1.2,0)  {$\alert{\ttt}$};
        \path       
    	node at (1.2,-0.5) {$\alert{\ccc}$};
        \pause
    	\path 
    	node at (1.5,0)  {$\alert{\gga}$};
        \path        
    	node at (1.5,-0.5) {$\alert{\ttt}$}
        node at (0,-1) {$Dist(\alpha,\beta) = 5$};
     
    \end{tikzpicture}    
	
\end{frame}


\begin{frame}[fragile]
	\frametitle{Solução Computacional}
	\textbf{Problema do \textit{Primer} com $k$ Diferenças}: 
    \begin{itemize}
    	\item \textbf{Entrada}: Duas sequências $\alpha$ e $\beta$ e um inteiro $k > 0$;
        \item \textbf{Objetivo}: Encontrar para cada índice $j$ de $\alpha$, o menor segmento $\gamma$ de $\alpha$ que se inicia em $j$ com distância de pelo menos $k$ com relação a qualquer segmento de $\beta$.
    \end{itemize}
\end{frame}

%% Começando aqui os exemplos
\begin{frame}[fragile]
	\frametitle{Exemplos}
	\begin{itemize}
		\item \textbf{Entrada}: $\alpha = AAGGAA$, $\beta = AAAGGA$ e $k = 2$
	\end{itemize}
    
    \begin{center}
     	\begin{tikzpicture}
        	\path 
            node at (0,0)    		{$\aaa$}
            node at (0.3,0)  		{$\aaa$}
            node at (0.6,0)  		{$\gga$}
            node at (0.9,0)  		{$\gga$}
            node at (1.2,0)    		{$\aaa$}
            node at (1.5,0)  		{$\aaa$};
            \path        
            node at (0,-0.5) 		{$\aaa$}
            node at (0.3,-0.5) 		{$\aaa$}
            node at (0.6,-0.5) 		{$\aaa$}
            node at (0.9,-0.5) 		{$\gga$}
            node at (1.2,-0.5) 		{$\gga$}
            node at (1.5,-0.5) 		{$\aaa$};
 
		\end{tikzpicture}
	\end{center}
\end{frame}


%% Fim aqui exemplos primer

%% Exemplos Método 1



\begin{frame}[fragile]
	\frametitle{Exemplos}
	\begin{itemize}
		\item \textbf{Entrada}: $\alpha = AAGGAA$, $\beta = AAAGGA$ e $k = 2$
	\end{itemize}
    
    \begin{center}
     	\begin{tikzpicture}
        	\path 
            node at (0,0)    		{$\aaa$}
            node at (0.3,0)  		{$\aaa$}
            node at (0.6,0)  		{$\gga$}
            node at (0.9,0)  		{$\gga$}
            node at (1.2,0)    		{$\aaa$}
            node at (1.5,0)  		{$\aaa$};
            \path        
            node at (0,-0.5) 		{$\aaa$}
            node at (0.3,-0.5) 		{$\aaa$}
            node at (0.6,-0.5) 		{$\aaa$}
            node at (0.9,-0.5) 		{$\gga$}
            node at (1.2,-0.5) 		{$\gga$}
            node at (1.5,-0.5) 		{$\aaa$};
            \pause
        	\path 
            node at (0,0)    		{$\textcolor{blue}{\aaa}$};
            \path        
            node at (0,-0.5) 		{$\textcolor{blue}{\aaa}$};    
            \pause
        	\path 
            node at (0.3,0)  		{$\textcolor{blue}{\aaa}$};
            \path        
            node at (0.3,-0.5) 		{$\textcolor{blue}{\aaa}$};
            \pause
        	\path 
            node at (0.6,0)  		{$\alert{\gga}$};
            \path       
            node at (0.6,-0.5) 		{$\alert{\aaa}$};
            \pause
        	\path 
            node at (0.9,0)  		{$\textcolor{blue}{\gga}$};
            \path       
            node at (0.9,-0.5) 		{$\textcolor{blue}{\gga}$};
            \pause
        	\path 
            node at (1.2,0)  		{$\alert{\aaa}$};
            \path       
            node at (1.2,-0.5) 		{$\alert{\gga}$};
            \path
            node at (0,-1) 		{$5$};
		\end{tikzpicture}
	\end{center}
\end{frame}

\begin{frame}[fragile]
	\frametitle{Exemplos}
	\begin{itemize}
		\item \textbf{Entrada}: $\alpha = AAGGAA$, $\beta = AAAGGA$ e $k = 2$
	\end{itemize}
    
    \begin{center}
     	\begin{tikzpicture}
        	\path 
            node at (0,0)    		{$\aaa$}
            node at (0.3,0)  		{$\aaa$}
            node at (0.6,0)  		{$\gga$}
            node at (0.9,0)  		{$\gga$}
            node at (1.2,0)    		{$\aaa$}
            node at (1.5,0)  		{$\aaa$};
            \path        
            node at (-0.3,-0.5) 	{$\aaa$}
            node at (0,-0.5) 		{$\aaa$}
            node at (0.3,-0.5) 		{$\aaa$}
            node at (0.6,-0.5) 		{$\gga$}
            node at (0.9,-0.5) 		{$\gga$}
            node at (1.2,-0.5) 		{$\aaa$};
            \pause
        	\path 
            node at (0,0)    		{$\textcolor{blue}{\aaa}$};
            \path        
            node at (0,-0.5) 		{$\textcolor{blue}{\aaa}$};    
            \pause
        	\path 
            node at (0.3,0)  		{$\textcolor{blue}{\aaa}$};
            \path        
            node at (0.3,-0.5) 		{$\textcolor{blue}{\aaa}$};
            \pause
        	\path 
            node at (0.6,0)  		{$\textcolor{blue}{\gga}$};
            \path       
            node at (0.6,-0.5) 		{$\textcolor{blue}{\gga}$};
            \pause
        	\path 
            node at (0.9,0)  		{$\textcolor{blue}{\gga}$};
            \path       
            node at (0.9,-0.5) 		{$\textcolor{blue}{\gga}$};
            \pause
        	\path 
            node at (1.2,0)  		{$\textcolor{blue}{\aaa}$};
            \path       
            node at (1.2,-0.5) 		{$\textcolor{blue}{\aaa}$};
            \path
            node at (0,-1) 		{$6$};
		\end{tikzpicture}
	\end{center}
\end{frame}

\begin{frame}[fragile]
	\frametitle{Exemplos}
	\begin{itemize}
		\item \textbf{Entrada}: $\alpha = AAGGAA$, $\beta = AAAGGA$ e $k = 2$
	\end{itemize}
    
    \begin{center}
     	\begin{tikzpicture}
        	\path 
            node at (-0.3,0)    	{$\aaa$}
            node at (0,0)  			{$\aaa$}
            node at (0.3,0)  		{$\gga$}
            node at (0.6,0)  		{$\gga$}
            node at (0.9,0)    		{$\aaa$}
            node at (1.2,0)  		{$\aaa$};
            \path        
            node at (0,-0.5) 		{$\aaa$}
            node at (0.3,-0.5) 		{$\aaa$}
            node at (0.6,-0.5) 		{$\aaa$}
            node at (0.9,-0.5) 		{$\gga$}
            node at (1.2,-0.5) 		{$\gga$}
            node at (1.5,-0.5) 		{$\aaa$};
            \path
            node at (-0.3,-1) 		{$6$};
            \pause
        	\path 
            node at (0,0)  			{$\textcolor{blue}{\aaa}$};
            \path        
            node at (0,-0.5) 		{$\textcolor{blue}{\aaa}$};   
            \pause
        	\path 
            node at (0.3,0)  			{$\alert{\gga}$};
            \path        
            node at (0.3,-0.5) 		{$\alert{\aaa}$};   
            \pause
        	\path 
            node at (0.6,0)  			{$\alert{\gga}$};
            \path        
            node at (0.6,-0.5) 		{$\alert{\aaa}$}; 
            \path
            node at (0, -1) {$3$};           
		\end{tikzpicture}
	\end{center}
\end{frame}

\begin{frame}[fragile]
	\frametitle{Exemplos}
	\begin{itemize}
		\item \textbf{Entrada}: $\alpha = AAGGAA$, $\beta = AAAGGA$ e $k = 2$
	\end{itemize}
    
    \begin{center}
     	\begin{tikzpicture}
        	\path 
            node at (-0.3,0)    	{$\aaa$}
            node at (0,0)  			{$\aaa$}
            node at (0.3,0)  		{$\gga$}
            node at (0.6,0)  		{$\gga$}
            node at (0.9,0)    		{$\aaa$}
            node at (1.2,0)  		{$\aaa$};
            \path        
            node at (-0.3,-0.5) 	{$\aaa$}
            node at (0,-0.5) 		{$\aaa$}
            node at (0.3,-0.5) 		{$\aaa$}
            node at (0.6,-0.5) 		{$\gga$}
            node at (0.9,-0.5) 		{$\gga$}
            node at (1.2,-0.5) 		{$\aaa$};
            \path
            node at (-0.3,-1) 		{$6$};
            \pause
        	\path 
            node at (0,0)  			{$\textcolor{blue}{\aaa}$};
            \path        
            node at (0,-0.5) 		{$\textcolor{blue}{\aaa}$};   
            \pause
        	\path 
            node at (0.3,0)  		{$\alert{\gga}$};
            \path        
            node at (0.3,-0.5) 		{$\alert{\aaa}$};   
            \pause
        	\path 
            node at (0.6,0)  		{$\textcolor{blue}{\gga}$};
            \path        
            node at (0.6,-0.5) 		{$\textcolor{blue}{\gga}$};  
            \pause
        	\path 
            node at (0.9,0)  		{$\alert{\aaa}$};
            \path        
            node at (0.9,-0.5) 		{$\alert{\gga}$}; 
            \path
            node at (0,-1) 		{$4$}; 

		\end{tikzpicture}
	\end{center}
\end{frame}

\begin{frame}[fragile]
	\frametitle{Exemplos}
	\begin{itemize}
		\item \textbf{Entrada}: $\alpha = AAGGAA$, $\beta = AAAGGA$ e $k = 2$
	\end{itemize}
    
    \begin{center}
     	\begin{tikzpicture}
        	\path 
            node at (-0.3,0)    	{$\aaa$}
            node at (0,0)  			{$\aaa$}
            node at (0.3,0)  		{$\gga$}
            node at (0.6,0)  		{$\gga$}
            node at (0.9,0)    		{$\aaa$}
            node at (1.2,0)  		{$\aaa$};
            \path        
            node at (-0.6,-0.5) 	{$\aaa$}
            node at (-0.3,-0.5) 	{$\aaa$}
            node at (0,-0.5) 		{$\aaa$}
            node at (0.3,-0.5) 		{$\gga$}
            node at (0.6,-0.5) 		{$\gga$}
            node at (0.9,-0.5) 		{$\aaa$};
            \path
            node at (-0.3,-1) 		{$6$};
            \pause
        	\path 
            node at (0,0)  			{$\textcolor{blue}{\aaa}$};
            \path        
            node at (0,-0.5) 		{$\textcolor{blue}{\aaa}$};   
            \pause
        	\path 
            node at (0.3,0)  		{$\textcolor{blue}{\gga}$};
            \path        
            node at (0.3,-0.5) 		{$\textcolor{blue}{\gga}$};   
            \pause
        	\path 
            node at (0.6,0)  		{$\textcolor{blue}{\gga}$};
            \path        
            node at (0.6,-0.5) 		{$\textcolor{blue}{\gga}$}; 
            \pause
        	\path 
            node at (0.9,0)  		{$\textcolor{blue}{\aaa}$};
            \path        
            node at (0.9,-0.5) 		{$\textcolor{blue}{\aaa}$}; 
            \path
            node at (0, -1) {$5$};
   		\end{tikzpicture}
	\end{center}
\end{frame}

\begin{frame}[fragile]
	\frametitle{Exemplos}
	\begin{itemize}
		\item \textbf{Entrada}: $\alpha = AAGGAA$, $\beta = AAAGGA$ e $k = 2$
	\end{itemize}
    
    \begin{center}
     	\begin{tikzpicture}
        	\path 
            node at (0,0)    		{$\aaa$}
            node at (0.3,0)  		{$\aaa$}
            node at (0.6,0)  		{$\gga$}
            node at (0.9,0)  		{$\gga$}
            node at (1.2,0)    		{$\aaa$}
            node at (1.5,0)  		{$\aaa$};
            \path        
            node at (0,-0.5) 		{$\aaa$}
            node at (0.3,-0.5) 		{$\aaa$}
            node at (0.6,-0.5) 		{$\aaa$}
            node at (0.9,-0.5) 		{$\gga$}
            node at (1.2,-0.5) 		{$\gga$}
            node at (1.5,-0.5) 		{$\aaa$};
            \path        
            node at (0,-1) 			{$6$}
            node at (0.3,-1) 		{$5$}
            node at (0.6,-1) 		{$4$}
            node at (0.9,-1) 		{$4$}
            node at (1.2,-1) 		{$3$}
            node at (1.5,-1) 		{$2$};
   		\end{tikzpicture}
	\end{center}
\end{frame}
%% terminando aqui os exemplos


\begin{frame}[fragile]
	\frametitle{Método 1}
    \begin{itemize}
    	\item Funciona exatamente como o exemplo anterior;
        \item $O(n^2)$ pares;
        \item $O(n)$ para percorrer toda cadeia;
        \item $O(n^3)$ para computação.
    \end{itemize}    
\end{frame}

\begin{frame}[fragile]
	\frametitle{Método 2}
	\begin{itemize}
		\item Podemos melhorar?
    	\item Podemos aproveitar informações?
        \pause
        \item SIM.
        \item Como?
	\end{itemize}
\end{frame}

%%% começando aqui

\begin{frame}[fragile]
	\frametitle{Método 2}
    
    \begin{center}
    	Alinhamento: \\
        
	    $(pos_v,pos_u) \in \{(0,l)|l=0 \ldots |u|\} \cup \{(l,0)|l=0 \ldots |v|\} $   
        \begin{itemize}
        	\centering
            \item \textbf{Entrada}: $\alpha = AAGGAA$, $\beta = AAAGGA$ e $k = 2$
        \end{itemize}        
        
     	\begin{tikzpicture}
        	\path 
            node at (0,0)    		{$\aaa$}
            node at (0.3,0)  		{$\aaa$}
            node at (0.6,0)  		{$\gga$}
            node at (0.9,0)  		{$\gga$}
            node at (1.2,0)    		{$\aaa$}
            node at (1.5,0)  		{$\aaa$};
            \path        
            node at (-1.5,-0.5) 	{$\aaa$}
            node at (-1.2,-0.5) 	{$\aaa$}
            node at (-0.9,-0.5) 	{$\aaa$}
            node at (-0.6,-0.5) 	{$\gga$}
            node at (-0.3,-0.5) 	{$\aaa$}
            node at (0,-0.5) 		{$\aaa$};
            
		\end{tikzpicture}
    \end{center}
\end{frame}

\begin{frame}[fragile]
	\frametitle{Método 2}
    
    \begin{center}
    	Alinhamento: \\
        
	    $(pos_v,pos_u) \in \{(0,l)|l=0 \ldots |u|\} \cup \{(l,0)|l=0 \ldots |v|\} $   
        \begin{itemize}
        	\centering
            \item \textbf{Entrada}: $\alpha = AAGGAA$, $\beta = AAAGGA$ e $k = 2$
        \end{itemize}        
        
     	\begin{tikzpicture}
        	\path 
            node at (0,0)    		{$\textcolor{blue}{\aaa}$}
            node at (0.3,0)  		{$\aaa$}
            node at (0.6,0)  		{$\gga$}
            node at (0.9,0)  		{$\gga$}
            node at (1.2,0)    		{$\aaa$}
            node at (1.5,0)  		{$\aaa$};
            \path        
            node at (-1.5,-0.5) 	{$\aaa$}
            node at (-1.2,-0.5) 	{$\aaa$}
            node at (-0.9,-0.5) 	{$\aaa$}
            node at (-0.6,-0.5) 	{$\gga$}
            node at (-0.3,-0.5) 	{$\gga$}
            node at (0,-0.5) 		{$\textcolor{blue}{\aaa}$};
            \path
			node at (0, -1) {2};            
		\end{tikzpicture}
    \end{center}
\end{frame}

\begin{frame}[fragile]
	\frametitle{Método 2}
    
    \begin{center}
    	Alinhamento: \\
        
	    $(pos_v,pos_u) \in \{(0,l)|l=0 \ldots |u|\} \cup \{(l,0)|l=0 \ldots |v|\} $   
        \begin{itemize}
        	\centering
            \item \textbf{Entrada}: $\alpha = AAGGAA$, $\beta = AAAGGA$ e $k = 2$
        \end{itemize}        
        
     	\begin{tikzpicture}
        	\path 
            node at (0,0)    		{$\alert{\aaa}$}
            node at (0.3,0)  		{$\textcolor{blue}{\aaa}$}
            node at (0.6,0)  		{$\gga$}
            node at (0.9,0)  		{$\gga$}
            node at (1.2,0)    		{$\aaa$}
            node at (1.5,0)  		{$\aaa$};
            \path        
            node at (-1.2,-0.5) 	{$\aaa$}
            node at (-0.9,-0.5) 	{$\aaa$}
            node at (-0.6,-0.5) 	{$\aaa$}
            node at (-0.3,-0.5) 	{$\gga$}
            node at (0,-0.5) 		{$\alert{\gga}$}
            node at (0.3,-0.5) 		{$\textcolor{blue}{\aaa}$};
            \path
            node at (0, -1) 	{3}
			node at (0.3, -1) 	{2};            
		\end{tikzpicture}
    \end{center}
\end{frame}

\begin{frame}[fragile]
	\frametitle{Método 2}
    
    \begin{center}
    	Alinhamento: \\
        
	    $(pos_v,pos_u) \in \{(0,l)|l=0 \ldots |u|\} \cup \{(l,0)|l=0 \ldots |v|\} $   
        \begin{itemize}
        	\centering
            \item \textbf{Entrada}: $\alpha = AAGGAA$, $\beta = AAAGGA$ e $k = 2$
        \end{itemize}        
        
     	\begin{tikzpicture}
        	\path 
            node at (0,0)    		{$\alert{\aaa}$}
            node at (0.3,0)  		{$\alert{\aaa}$}
            node at (0.6,0)  		{$\alert{\gga}$}
            node at (0.9,0)  		{$\gga$}
            node at (1.2,0)    		{$\aaa$}
            node at (1.5,0)  		{$\aaa$};
            \path        
            node at (-0.9,-0.5) 	{$\aaa$}
            node at (-0.6,-0.5) 	{$\aaa$}
            node at (-0.3,-0.5) 	{$\aaa$}
            node at (0,-0.5) 		{$\alert{\gga}$}
            node at (0.3,-0.5) 		{$\alert{\gga}$}
            node at (0.6,-0.5) 		{$\alert{\aaa}$};
            \path
            node at (0, -1) 	{3}
			node at (0.3, -1) 	{2}
			node at (0.6, -1) 	{2};            
		\end{tikzpicture}
    \end{center}
\end{frame}

\begin{frame}[fragile]
	\frametitle{Método 2}
    
    \begin{center}
    	Alinhamento: \\
        
	    $(pos_v,pos_u) \in \{(0,l)|l=0 \ldots |u|\} \cup \{(l,0)|l=0 \ldots |v|\} $   
        \begin{itemize}
        	\centering
            \item \textbf{Entrada}: $\alpha = AAGGAA$, $\beta = AAAGGA$ e $k = 2$
        \end{itemize}        
        
     	\begin{tikzpicture}
        	\path 
            node at (0,0)    		{$\textcolor{blue}{\aaa}$}
            node at (0.3,0)  		{$\alert{\aaa}$}
            node at (0.6,0)  		{$\textcolor{blue}{\gga}$}
            node at (0.9,0)  		{$\alert{\gga}$}
            node at (1.2,0)    		{$\aaa$}
            node at (1.5,0)  		{$\aaa$};
            \path        
            node at (-0.6,-0.5) 	{$\aaa$}
            node at (-0.3,-0.5) 	{$\aaa$}
            node at (0,-0.5) 		{$\textcolor{blue}{\aaa}$}
            node at (0.3,-0.5) 		{$\alert{\gga}$}
            node at (0.6,-0.5) 		{$\textcolor{blue}{\gga}$}
            node at (0.9,-0.5) 		{$\alert{\aaa}$};
            \path
            node at (0, -1) 	{4}
            node at (0.3, -1) 	{3}
			node at (0.6, -1) 	{3}
			node at (0.9, -1) 	{2};            
		\end{tikzpicture}
    \end{center}
\end{frame}

\begin{frame}[fragile]
	\frametitle{Método 2}
    
    \begin{center}
    	Alinhamento: \\
        
	    $(pos_v,pos_u) \in \{(0,l)|l=0 \ldots |u|\} \cup \{(l,0)|l=0 \ldots |v|\} $   
        \begin{itemize}
        	\centering
            \item \textbf{Entrada}: $\alpha = AAGGAA$, $\beta = AAAGGA$ e $k = 2$
        \end{itemize}        
        
     	\begin{tikzpicture}
        	\path 
            node at (0,0)    		{$\textcolor{blue}{\aaa}$}
            node at (0.3,0)  		{$\textcolor{blue}{\aaa}$}
            node at (0.6,0)  		{$\textcolor{blue}{\gga}$}
            node at (0.9,0)  		{$\textcolor{blue}{\gga}$}
            node at (1.2,0)    		{$\textcolor{blue}{\aaa}$}
            node at (1.5,0)  		{$\aaa$};
            \path        
            node at (-0.3,-0.5) 	{$\aaa$}
            node at (0,-0.5) 		{$\textcolor{blue}{\aaa}$}
            node at (0.3,-0.5) 		{$\textcolor{blue}{\aaa}$}
            node at (0.6,-0.5) 		{$\textcolor{blue}{\gga}$}
            node at (0.9,-0.5) 		{$\textcolor{blue}{\gga}$}
            node at (1.2,-0.5) 		{$\textcolor{blue}{\aaa}$};
            \path
            node at (0, -1) 	{6}
			node at (0.3, -1) 	{5}
            node at (0.6, -1) 	{4}
			node at (0.9, -1) 	{3}
			node at (1.2, -1) 	{2};            
		\end{tikzpicture}
    \end{center}
\end{frame}

\begin{frame}[fragile]
	\frametitle{Método 2}
    
    \begin{center}
    	Alinhamento: \\
        
	    $(pos_v,pos_u) \in \{(0,l)|l=0 \ldots |u|\} \cup \{(l,0)|l=0 \ldots |v|\} $   
        \begin{itemize}
        	\centering
            \item \textbf{Entrada}: $\alpha = AAGGAA$, $\beta = AAAGGA$ e $k = 2$
        \end{itemize}        
        
     	\begin{tikzpicture}
        	\path 
            node at (0,0)    		{$\textcolor{blue}{\aaa}$}
            node at (0.3,0)  		{$\textcolor{blue}{\aaa}$}
            node at (0.6,0)  		{$\alert{\gga}$}
            node at (0.9,0)  		{$\textcolor{blue}{\gga}$}
            node at (1.2,0)    		{$\alert{\aaa}$}
            node at (1.5,0)  		{$\textcolor{blue}{\aaa}$};
            \path        
            node at (0,-0.5) 		{$\textcolor{blue}{\aaa}$}
            node at (0.3,-0.5) 		{$\textcolor{blue}{\aaa}$}
            node at (0.6,-0.5) 		{$\alert{\aaa}$}
            node at (0.9,-0.5) 		{$\textcolor{blue}{\gga}$}
            node at (1.2,-0.5) 		{$\alert{\gga}$}
            node at (1.5,-0.5) 		{$\textcolor{blue}{\aaa}$};
            \path
            node at (0, -1) 	{6}
			node at (0.3, -1) 	{5}
            node at (0.6, -1) 	{4}
			node at (0.9, -1) 	{4}
			node at (1.2, -1) 	{3}
            node at (1.5, -1)   {2};            
		\end{tikzpicture}
    \end{center}
\end{frame}

\begin{frame}[fragile]
	\frametitle{Método 2}
    
    \begin{center}
    	Alinhamento: \\
        
	    $(pos_v,pos_u) \in \{(0,l)|l=0 \ldots |u|\} \cup \{(l,0)|l=0 \ldots |v|\} $   
        \begin{itemize}
        	\centering
            \item \textbf{Entrada}: $\alpha = AAGGAA$, $\beta = AAAGGA$ e $k = 2$
        \end{itemize}        
        
     	\begin{tikzpicture}
        	\path 
            node at (0,0)    		{$\aaa$}
            node at (0.3,0)  		{$\textcolor{blue}{\aaa}$}
            node at (0.6,0)  		{$\alert{\gga}$}
            node at (0.9,0)  		{$\alert{\gga}$}
            node at (1.2,0)    		{$\alert{\aaa}$}
            node at (1.5,0)  		{$\alert{\aaa}$};
            \path        
            node at (0.3,-0.5) 		{$\textcolor{blue}{\aaa}$}
            node at (0.6,-0.5) 		{$\alert{\aaa}$}
            node at (0.9,-0.5) 		{$\alert{\aaa}$}
            node at (1.2,-0.5) 		{$\alert{\gga}$}
            node at (1.5,-0.5) 		{$\alert{\gga}$}
            node at (1.8,-0.5) 		{$\aaa$};
            \path
            node at (0, -1) 	{6}
			node at (0.3, -1) 	{5}
            node at (0.6, -1) 	{4}
			node at (0.9, -1) 	{4}
			node at (1.2, -1) 	{3}
            node at (1.5, -1)   {2};            
		\end{tikzpicture}
    \end{center}
\end{frame}

\begin{frame}[fragile]
	\frametitle{Método 2}
    
    \begin{center}
    	Alinhamento: \\
        
	    $(pos_v,pos_u) \in \{(0,l)|l=0 \ldots |u|\} \cup \{(l,0)|l=0 \ldots |v|\} $   
        \begin{itemize}
        	\centering
            \item \textbf{Entrada}: $\alpha = AAGGAA$, $\beta = AAAGGA$ e $k = 2$
        \end{itemize}        
        
     	\begin{tikzpicture}
        	\path 
            node at (0,0)    		{$\aaa$}
            node at (0.3,0)  		{$\aaa$}
            node at (0.6,0)  		{$\alert{\gga}$}
            node at (0.9,0)  		{$\alert{\gga}$}
            node at (1.2,0)    		{$\alert{\aaa}$}
            node at (1.5,0)  		{$\alert{\aaa}$};
            \path        
            node at (0.6,-0.5) 		{$\alert{\aaa}$}
            node at (0.9,-0.5) 		{$\alert{\aaa}$}
            node at (1.2,-0.5) 		{$\alert{\aaa}$}
            node at (1.5,-0.5) 		{$\alert{\gga}$}
            node at (1.8,-0.5) 		{$\gga$}
            node at (2.1,-0.5) 		{$\aaa$};
            \path
            node at (0, -1) 	{6}
			node at (0.3, -1) 	{5}
            node at (0.6, -1) 	{4}
			node at (0.9, -1) 	{4}
			node at (1.2, -1) 	{3}
            node at (1.5, -1)   {2};            
		\end{tikzpicture}
    \end{center}
\end{frame}

\begin{frame}[fragile]
	\frametitle{Método 2}
    
    \begin{center}
    	Alinhamento: \\
        
	    $(pos_v,pos_u) \in \{(0,l)|l=0 \ldots |u|\} \cup \{(l,0)|l=0 \ldots |v|\} $   
        \begin{itemize}
        	\centering
            \item \textbf{Entrada}: $\alpha = AAGGAA$, $\beta = AAAGGA$ e $k = 2$
        \end{itemize}        
        
     	\begin{tikzpicture}
        	\path 
            node at (0,0)    		{$\aaa$}
            node at (0.3,0)  		{$\aaa$}
            node at (0.6,0)  		{$\gga$}
            node at (0.9,0)  		{$\alert{\gga}$}
            node at (1.2,0)    		{$\textcolor{blue}{\aaa}$}
            node at (1.5,0)  		{$\textcolor{blue}{\aaa}$};
            \path        
            node at (0.9,-0.5) 		{$\alert{\aaa}$}
            node at (1.2,-0.5) 		{$\textcolor{blue}{\aaa}$}
            node at (1.5,-0.5) 		{$\textcolor{blue}{\aaa}$}
            node at (1.8,-0.5) 		{$\gga$}
            node at (2.1,-0.5) 		{$\gga$}
            node at (2.4,-0.5) 		{$\aaa$};
            \path
            node at (0, -1) 	{6}
			node at (0.3, -1) 	{5}
            node at (0.6, -1) 	{4}
			node at (0.9, -1) 	{4}
			node at (1.2, -1) 	{3}
            node at (1.5, -1)   {2};            
		\end{tikzpicture}
    \end{center}
\end{frame}

\begin{frame}[fragile]
	\frametitle{Método 2}
    
    \begin{center}
    	Alinhamento: \\
        
	    $(pos_v,pos_u) \in \{(0,l)|l=0 \ldots |u|\} \cup \{(l,0)|l=0 \ldots |v|\} $   
        \begin{itemize}
        	\centering
            \item \textbf{Entrada}: $\alpha = AAGGAA$, $\beta = AAAGGA$ e $k = 2$
        \end{itemize}        
        
     	\begin{tikzpicture}
        	\path 
            node at (0,0)    		{$\aaa$}
            node at (0.3,0)  		{$\aaa$}
            node at (0.6,0)  		{$\gga$}
            node at (0.9,0)  		{$\gga$}
            node at (1.2,0)    		{$\textcolor{blue}{\aaa}$}
            node at (1.5,0)  		{$\textcolor{blue}{\aaa}$};
            \path        
            node at (1.2,-0.5) 		{$\textcolor{blue}{\aaa}$}
            node at (1.5,-0.5) 		{$\textcolor{blue}{\aaa}$}
            node at (1.8,-0.5) 		{$\aaa$}
            node at (2.1,-0.5) 		{$\gga$}
            node at (2.4,-0.5) 		{$\gga$}
            node at (2.7,-0.5) 		{$\aaa$};
            \path
            node at (0, -1) 	{6}
			node at (0.3, -1) 	{5}
            node at (0.6, -1) 	{4}
			node at (0.9, -1) 	{4}
			node at (1.2, -1) 	{3}
            node at (1.5, -1)   {2};            
		\end{tikzpicture}
    \end{center}
\end{frame}

\begin{frame}[fragile]
	\frametitle{Método 2}
    
    \begin{center}
    	Alinhamento: \\
        
	    $(pos_v,pos_u) \in \{(0,l)|l=0 \ldots |u|\} \cup \{(l,0)|l=0 \ldots |v|\} $   
        \begin{itemize}
        	\centering
            \item \textbf{Entrada}: $\alpha = AAGGAA$, $\beta = AAAGGA$ e $k = 2$
        \end{itemize}        
        
     	\begin{tikzpicture}
        	\path 
            node at (0,0)    		{$\aaa$}
            node at (0.3,0)  		{$\aaa$}
            node at (0.6,0)  		{$\gga$}
            node at (0.9,0)  		{$\gga$}
            node at (1.2,0)    		{$\aaa$}
            node at (1.5,0)  		{$\textcolor{blue}{\aaa}$};
            \path        
            node at (1.5,-0.5) 		{$\textcolor{blue}{\aaa}$}
            node at (1.8,-0.5) 		{$\aaa$}
            node at (2.1,-0.5) 		{$\aaa$}
            node at (2.4,-0.5) 		{$\gga$}
            node at (2.7,-0.5) 		{$\gga$}
            node at (3.0,-0.5) 		{$\aaa$};
            \path
            node at (0, -1) 	{6}
			node at (0.3, -1) 	{5}
            node at (0.6, -1) 	{4}
			node at (0.9, -1) 	{\textcolor{red}{4}}
			node at (1.2, -1) 	{\textcolor{red}{3}}
            node at (1.5, -1)   {\textcolor{red}{2}};            
		\end{tikzpicture}
    \end{center}
\end{frame}

%%% terminando aqui



\begin{frame}[fragile]
	\frametitle{Método 2}
    \begin{itemize}
    	\item Funciona exatamente como o exemplo anterior;
        \item $O(2n)$ pares;
        \item $O(n)$ para percorrer toda cadeia;
        \item $O(n^2)$ para computação.
    \end{itemize}    
\end{frame}

\begin{frame}[fragile]
	\frametitle{Resultados}
	\begin{figure}[!hbt]
		\centering
		\includegraphics[scale=.75]{k30.eps}
		\caption{\textit{Gráfico de tempo de execução das sequências diferentes, geradas aleatoriamente, }$k = 30.$}
	\end{figure}
\end{frame}

\begin{frame}[fragile]
	\frametitle{Resultados}
	\begin{figure}[!hbt]
		\centering
		\includegraphics[scale=.75]{k300.eps}
		\caption{\textit{Gráfico de tempo de execução das sequências diferentes, geradas aleatoriamente, }$k = 300.$}
	\end{figure}
\end{frame}

\begin{frame}[fragile]
	\frametitle{Resultados}
	\begin{figure}[!hbt]
		\centering
		\includegraphics[scale=.75]{k600.eps}
		\caption{\textit{Gráfico de tempo de execução das sequências diferentes, geradas aleatoriamente, }$k = 600.$}
	\end{figure}
\end{frame}

\begin{frame}[fragile]
	\frametitle{Resultados}
	\begin{figure}[!hbt]
		\centering
		\includegraphics[scale=.75]{graficoIguais.eps}
		\caption{Gráfico de tempo de execução das sequências iguais.}
	\end{figure}
\end{frame}

\begin{frame}[fragile]
	\frametitle{Resultados}
    \begin{figure}[!hbt]
    \centering
        \includegraphics[scale=.75]{hsarhgdig.eps}
        \caption{Gráfico de tempo de execução da sequência hsarhgdig (tam. 4398), $k = 30, 300$, e $600$}
    \end{figure}
\end{frame}

\begin{frame}[fragile]
	\frametitle{Resultados}
    \begin{figure}[!hbt]
    \centering
        \includegraphics[scale=.75]{hsankr10.eps}
        \caption{Gráfico de tempo de execução da sequência hsankr10 (tam. 38530), $k = 30, 300$, e $600$}
    \end{figure}
\end{frame}

\begin{frame}[fragile]
	\frametitle{Resultados}
    \begin{figure}[!hbt]
    \centering
        \includegraphics[scale=.75]{hsaff4.eps}
    \caption{Gráfico de tempo de execução da sequência hsaff4 (tam. 90284), $k = 30, 300$, e $600$}
    \end{figure}
\end{frame}

\begin{frame}[fragile]
	\frametitle{Podemos comparar esse trabalho com o do Jean?}
    
    \begin{itemize}
    	\item "NÃO".
        \centering
    \end{itemize}

\end{frame}

\begin{frame}[fragile]
	\frametitle{Motivo}
	\begin{itemize}
		\item \textbf{Entrada}: $\alpha = AAGGAA$, $\beta = AAAGGA$ e $k = 2$
	\end{itemize}
    
    \textbf{Hamming}:
    \begin{center}
    	\begin{tikzpicture}
        	\path 
            node at (0,0)    	{$\aaa$}
            node at (0.3,0)  	{$\aaa$}
            node at (0.6,0)  	{$\gga$}
            node at (0.9,0)  	{$\gga$}
            node at (1.2,0)  	{$\aaa$}
            node at (1.5,0)  	{$\aaa$};
            \path        
            node at (0,-0.5) 	{$\aaa$}
            node at (0.3,-0.5) 	{$\aaa$}
            node at (0.6,-0.5) 	{$\aaa$}
            node at (0.9,-0.5) 	{$\gga$}
            node at (1.2,-0.5) 	{$\gga$}
            node at (1.5,-0.5) 	{$\aaa$};            
            \path        
            node at (0,-1) 		{$\textcolor{blue}{6}$}
            node at (0.3,-1) 	{$\textcolor{blue}{5}$}
            node at (0.6,-1) 	{$\textcolor{blue}{4}$}
            node at (0.9,-1) 	{$\textcolor{red}{4}$}
            node at (1.2,-1) 	{$\textcolor{red}{3}$}
            node at (1.5,-1) 	{$\textcolor{red}{2}$};  
		\end{tikzpicture} 
	\end{center}
    
    \textbf{Edição}: Não foi encontrado nenhuma ocorrência com pelo menos 2 diferença(s).
    
\end{frame}

\plain{Questões?}

\begin{frame}[fragile]
	\frametitle{Referências}
    \begin{thebibliography}{9}
    	\bibitem{1} Dobre, J. A. (2017). O Problema da Seleção de Segmentos Específicos: algoritmos e aplicações.
        
       	\bibitem{2} Alberts, B., Johnson, A., Lewis, J., Raff, M., Roberts, K., and Walter, P. (2009). Biologia
		\bibitem{3} do Nascimento, F. M. S. (2008). Aplicação da técnica pcr para detecção de bactérias
potencialmente patogênicas em um sistema uasb-lagoas de polimento para tratamento
de esgoto doméstico.
		\bibitem{4} Gusfield, D. (1997). Algorithms on Strings, Trees, and Sequences: Computer Science and
Computational Biology. EBL-Schweitzer. Cambridge University Press.
    \end{thebibliography}
\end{frame}

\begin{frame}[fragile]
	\frametitle{Referências}
    \begin{thebibliography}{9}
		\bibitem{5} Landau, G. M. and Vishkin, U. (1989). Fast parallel and serial approximate string matching.
Journal of algorithms, 10(2):157–169.
		\bibitem{6} Maaß, M. G. et al. (2003). A fast algorithm for the inexact characteristic string problem.
Technical report, Technical Report TUM-I0312, Fakultät für Informatik, TU München.
		\bibitem{7} Ribeiro, M. C. M. (2009). Genética Molelcular.
    \end{thebibliography}
\end{frame}

\end{document}